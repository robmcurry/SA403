% LaTeX Article Template
\documentclass[12pt]{article}
%% Other packages
\usepackage{amsmath}
\usepackage{amsthm}
\usepackage{titlesec}
\usepackage{soul}
\usepackage{tikz}
\usepackage{tikz-3dplot}
\usepackage{amssymb}
\usepackage{multicol}
\usepackage{float}
\usepackage{calc}
\usepackage{fancybox}
\usepackage{array}
\usepackage[shortlabels]{enumitem}
\usepackage{framed}
\usepackage{hyperref}
\newcolumntype{L}[1]{>{\raggedright\let\newline\\\arraybackslash\hspace{0pt}}m{#1}}
\newcolumntype{C}[1]{>{\centering\let\newline\\\arraybackslash\hspace{0pt}}m{#1}}
\newcolumntype{R}[1]{>{\raggedleft\let\newline\\\arraybackslash\hspace{0pt}}m{#1}}


%% Margins
\usepackage{geometry}
\geometry{verbose,letterpaper,tmargin=1in,bmargin=1in,lmargin=1in,rmargin=1in}

\newcommand{\menuchoice}[2]{{\ttfamily#1..#2}}
\newcommand{\dotdot}{..}

\usepackage{graphicx}

% Array vertical and horizontal stretch
% \def\arraystretch{1.5}%  1 is the default, change whatever you need
% \setlength{\tabcolsep}{12pt}

%\graphicspath{%
\graphicspath{{./figs/}}

%% Paragraph style settings
\setlength{\parskip}{\medskipamount}
\setlength{\parindent}{0pt}

%% Change itemize bullets
\renewcommand{\labelitemi}{$\bullet$}
\renewcommand{\labelitemii}{$\circ$}
\renewcommand{\labelitemiii}{$\diamond$}
\renewcommand{\labelitemiv}{$\cdot$}

%% Shrink section fonts
\titleformat*{\section}{\large\bf}
\titleformat*{\subsection}{\normalsize\it}
\titleformat*{\subsubsection}{\normalsize\bf}

% %% Compress the spacing around section titles
\titlespacing*{\section}{0pt}{1.5ex}{0.75ex}
\titlespacing*{\subsection}{0pt}{1ex}{0.5ex}
\titlespacing*{\subsubsection}{0pt}{1ex}{0.5ex}

%% amsthm settings
\theoremstyle{definition}
\newtheorem{problem}{Problem}
\newtheorem{example}{Example}
\newtheorem{mydef}{Definition}

%% Answer box macros
%% \answerbox{alignment}{width}{height}
\newcommand{\answerbox}[3]{%
  \fbox{%
    \begin{minipage}[#1]{#2}
      \hfill\vspace{#3}
    \end{minipage}
  }
}

%% \answerboxfull{alignment}{height}
\newcommand{\answerboxfull}[2]{%
  \answerbox{#1}{\textwidth}{#2} 
}

%% \answerboxone{alignment}{height} -- for first-level bullet
\newcommand{\answerboxone}[2]{%
  \answerbox{#1}{6.15in}{#2} 
}

%% \answerboxtwo{alignment}{height} -- for second-level bullet
\newcommand{\answerboxtwo}[2]{%
  \answerbox{#1}{5.8in}{#2}
}

%% \graphbox{xmin}{xmax}{ymin}{ymax}{scale}
\newcommand{\graphbox}[5]%[-5, 5, -5, 5, 0.33]
{
\begin{tikzpicture}
     [>=latex,scale=#5]
     
     % Coordinate axes
     \draw [->,very thick] (#1, 0) -- (#2, 0) node[right] {$x$};
     \draw [->,very thick] (0, #3) -- (0, #4) node[above] {$y$};
     
     % Grid
     \draw[step=1cm,thick,dotted] (#1,#3) grid (#2,#4);
   \end{tikzpicture}
   }


%% Redefine maketitle
\makeatletter
\renewcommand{\maketitle}{
  \noindent SA403 -- Networks \\

  \begin{center}\Large{\textbf{\@title}}\end{center}
}
\makeatother

% Set the beginning of a LaTeX document
\begin{document}

%\graphbox{-10}{3}{-5}{10}

\title{Lesson: Introducing Notation}

%\graphbox[10][10]

\maketitle


%\section*{Notes}

Book acknowledgment:
\begin{itemize}
	\item[] ``Linear Programming and Network Flows" by Bazaraa, Jarvis, and Sherali. \emph{Fourth Edition}
\end{itemize}
\section*{Goals}
\begin{itemize}
\item  Introduce (or re-introduce) network/graph notation.
\item Make sure that we are on the same page with definitions and notation.
\end{itemize}

\section{Undirected Graphs}

Let $G = (V,E)$ be an \textbf{undirected graph}.
\begin{itemize}
	\item[] $V$ is a finite set of nodes (or vertices) with $n:= |V|$.
	\item[] $E$ is a collection of unordered pairs (edges) of elements of $V$ with $m:=|E|$.
\end{itemize}
 
\subsection*{Attributes}

\begin{itemize}
	\item $G$ has \emph{weights} on edges and/or nodes. 
	\begin{itemize}
		\item For edges, $c_{ij}, \ \forall (i,j) \in E$. 
		\item For nodes, $w_i, \ \forall i \in V$.
	\end{itemize}
	\item Let $V= \{1,2, \dots, n\}$.
	\item Let the edges in $E$ have the form $(i,j), \forall i,j \in V$
	\begin{itemize}
		\item For the sake of simplicity, we assume that edges $(i,j)$ and $(j,i)$ are equivalent. Thus, we assume that $E$ only consists of edges $(i,j)$ for which $i<j$.
	\end{itemize}
	\item Two nodes $i$ and $j$ are said to be \emph{adjacent} when there exists an edge $(i,j)$ that connects them. 
	\item Two edges are referred to as \emph{adjacent} if they have an node in common.
	\item The number of edges incident on a vertex $v$ is called the \emph{degree} of the vertex, and is usually denoted by $d_G(v)$. 
	\item Graph $G$ is \emph{complete}  if it contains all possible edges. (When $E = \{(i,j): i,j \in V, i < j\}$. 
	\item Graph $G' = (V', E')$ is a \emph{subgraph} of $G$ if $V' \subseteq V$ and $E' \subseteq E$. Additionally, for $G'$ to be a subgraph of $G$, then for all edges $(i,j) \in E'$, then both vertices $i$ and $j$ must belong to $V'$.
	\item A \emph{path} in $G$ is a sequence of consecutive edges $e_1, e_2, \dots, e_k \in E$, in which $e_1 = (v_1, v_2), \ e_2 = (v_2, v_3), \dots, e_k = (v_k, v_{k+1})$. The path connects $v_1$ and $v_{k+1}$, visiting the intermediate vertices $v_2, \ v_3, \dots, v_k$. 
	\begin{itemize}
		\item The path is a \emph{cycle} if $v_1 = v_{k+1}$. 
		\item A path is \emph{elementary} if no edge is used twice.
		\item The path is referred to as \emph{simple} if no node is visited twice.
		\item An elementary path is defined as an \emph{Eulerian} path if it visits every edge in $E$ once and only once. 
		\item A simple path is said to be \emph{Hamiltonian} if it visits every vertex in $V$ once and only once.
	\end{itemize}
	\item Vertex $v$ is \emph{connected} to vertex $w$ if there exists a path connecting them. 
	\item A graph $G$ is referred to as \emph{connected} if all vertices $v$ and $w$ in $V$ are connected.
	\item  A \emph{cut} in $G$ is a set of edges of the type:
		\begin{align*}
			\gamma_G(S):= \{(i,j) \in E: |S \cap \{i,j\} | = 1\}
		\end{align*}
	in which $S$ is the subset of vertices that induces the cut (i.e., the cut contains all edges with one endpoint in $S$ and the other in $V \setminus S$). 
	%\item One can easily verify that $G$ is connected if and only if $\gamma(S) \ne \emptyset$ for all $\emptyset \subset S \subset V$. Given two vertices $s$ and $t$, there exist $k$ edge-disjoint paths connecting them if and only if $|\gamma(S)| \ge k$ for all $S \subset V$ such that $s \in S$, $t \notin S$.
	\item A partial graph $G' = (V, E'$ of $G$ is called a \emph{forest} if it is \emph{acyclic}--it does not contain a cycle. 
	\begin{itemize}
		\item A forest is \emph{maximal} if every edge in $E \setminus E'$ forms a cycle with the edges in $E'$. Therefore, $G'$ and $G$ have the same connected components. 
	\end{itemize}
	\item A maximal connected forest, if it exists, is called a \emph{spanning tree}. Every tree has exactly $|V| - 1$ edges. Graph $G$ contains a tree if and only if $G$ is connected. 
	\item A graph is said to be \emph{bipartite} if there exists a partition $(V_1, V_2)$ of $V$ such that each e3dge $(i,j) \in E$ connects a vertex $i \in V_1$ to a vertex $j \in V_2$. 
	\begin{itemize}
		\item Graph $G$ is bipartite if and only if it does not contain any cycles visiting an odd number of vertices.
	\end{itemize}
	%\item A \emph{clique} is a subgraph $G' = (V',E')$ of $G$ in which every pair of vertices in $V'$ is connected by an edge.
	%\item A \emph{stable} set of $G$ is a subgraph $G' = (V', E')$ induced by $V'$ such that $E' = \emptyset$. 
\end{itemize}





\section{Directed Graphs}
A \emph{directed graph} is a pair $G = (V, A)$.

\begin{itemize}
	%\item $V$ is a finite set of vertices.
	%\item $A$ is a family of \emph{arcs} $(i,j) \in A$.
	%\item The order in which the nodes $i$ and $j$ appear is relevant; thus, $(i,j) \ne (j,i)$. 
	%\begin{itemize}
		\item In this case, we say that arc $(i,j)$ leaves node $i$ and enters node $j$.
		\item Nodes $i$ and $j$ are often referred to as the out-going and in-coming nodes, respectively. 
	%\end{itemize}
	%\item A \emph{directed path} is a sequences of arcs $a_1, a_2, \dots, a_k$ of consecutive arcs of the type $a_1 = (v_1, v_2), \ (v_2,v_3), \dots, (v_k, v_{k+1}$. 
\end{itemize}








\end{document}
