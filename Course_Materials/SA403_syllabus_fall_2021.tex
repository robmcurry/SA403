% LaTeX Article Template
\documentclass[12pt]{article}
%% Other packages
\usepackage{amsmath}
\usepackage{amsthm}
\usepackage{titlesec}
\usepackage{soul}
\usepackage{tikz}
\usepackage{tikz-3dplot}
\usepackage{amssymb}
\usepackage{multicol}
\usepackage{float}
\usepackage{calc}
\usepackage{fancybox}
\usepackage{array}
\usepackage[shortlabels]{enumitem}
\usepackage{framed}
\usepackage{hyperref}
\usepackage{longtable}

\newcolumntype{L}[1]{>{\raggedright\let\newline\\\arraybackslash\hspace{0pt}}m{#1}}
\newcolumntype{C}[1]{>{\centering\let\newline\\\arraybackslash\hspace{0pt}}m{#1}}
\newcolumntype{R}[1]{>{\raggedleft\let\newline\\\arraybackslash\hspace{0pt}}m{#1}}


%% Margins
\usepackage{geometry}
\geometry{verbose,letterpaper,tmargin=1in,bmargin=1in,lmargin=1in,rmargin=1in}

\newcommand{\menuchoice}[2]{{\ttfamily#1..#2}}
\newcommand{\dotdot}{..}

\usepackage{graphicx}

% Array vertical and horizontal stretch
% \def\arraystretch{1.5}%  1 is the default, change whatever you need
% \setlength{\tabcolsep}{12pt}

%\graphicspath{%
\graphicspath{{./figs/}}

%% Paragraph style settings
\setlength{\parskip}{\medskipamount}
\setlength{\parindent}{0pt}

%% Change itemize bullets
\renewcommand{\labelitemi}{$\bullet$}
\renewcommand{\labelitemii}{$\circ$}
\renewcommand{\labelitemiii}{$\diamond$}
\renewcommand{\labelitemiv}{$\cdot$}

%% Shrink section fonts
\titleformat*{\section}{\large\bf}
\titleformat*{\subsection}{\normalsize\it}
\titleformat*{\subsubsection}{\normalsize\bf}

% %% Compress the spacing around section titles
\titlespacing*{\section}{0pt}{1.5ex}{0.75ex}
\titlespacing*{\subsection}{0pt}{1ex}{0.5ex}
\titlespacing*{\subsubsection}{0pt}{1ex}{0.5ex}

%% amsthm settings
\theoremstyle{definition}
\newtheorem{problem}{Problem}
\newtheorem{example}{Example}
\newtheorem{mydef}{Definition}

%% Answer box macros
%% \answerbox{alignment}{width}{height}
\newcommand{\answerbox}[3]{%
  \fbox{%
    \begin{minipage}[#1]{#2}
      \hfill\vspace{#3}
    \end{minipage}
  }
}

%% \answerboxfull{alignment}{height}
\newcommand{\answerboxfull}[2]{%
  \answerbox{#1}{\textwidth}{#2} 
}

%% \answerboxone{alignment}{height} -- for first-level bullet
\newcommand{\answerboxone}[2]{%
  \answerbox{#1}{6.15in}{#2} 
}

%% \answerboxtwo{alignment}{height} -- for second-level bullet
\newcommand{\answerboxtwo}[2]{%
  \answerbox{#1}{5.8in}{#2}
}

%% \graphbox{xmin}{xmax}{ymin}{ymax}{scale}
\newcommand{\graphbox}[5]%[-5, 5, -5, 5, 0.33]
{
\begin{tikzpicture}
     [>=latex,scale=#5]
     
     % Coordinate axes
     \draw [->,very thick] (#1, 0) -- (#2, 0) node[right] {$x$};
     \draw [->,very thick] (0, #3) -- (0, #4) node[above] {$y$};
     
     % Grid
     \draw[step=1cm,thick,dotted] (#1,#3) grid (#2,#4);
   \end{tikzpicture}
   }


%% Redefine maketitle
\makeatletter
\renewcommand{\maketitle}{
  \noindent SA403 -- Graph \& Network Algorithms \hfill Fall 2021

  \begin{center}\large{\textbf{\@title}}\end{center}
}
\makeatother

% Set the beginning of a LaTeX document
\begin{document}

%\graphbox{-10}{3}{-5}{10}

\title{Syllabus}

%\graphbox[10][10]

\maketitle

\noindent \textbf{Course coordinator:}  Asst. Prof. Robert M. Curry  (rcurry@usna.edu)

\noindent \textbf{Textbook:} (None)

\noindent \textbf{Course description: } This course introduces graph algorithms for problems in network and combinatorial optimization. Topics include: minimum spanning trees, matchings, shortest paths, maximum flows and minimum cost flows. Students will also be expected to program algorithms on a computer.

\noindent \textbf{Course objectives:}  By the end of this course, students will be able to
\vspace{-2mm}
\begin{enumerate}[(i)]
	\item think critically and creatively;
	\item problem-solve;
	\item create, code, and analyze various network and graph algorithms;
	\item successfully collaborate and code in groups;
	\item clearly and concisely communicate the steps of various algorithms.
\end{enumerate}

\noindent \textbf{Approximate weekly course schedule:} 

\renewcommand\arraystretch{1.5}
\begin{longtable}{ll}
Week \hspace{.2in} & Topic \\
\hline 
\multicolumn{2}{l}{\textbf{\textit{Par 1: Foundational Algorithms}}}\\
1 & Introduction to Notation, Algorithms, and Software.  \\
2 & Graph Search Algorithms \\
3 & Minimum Spanning Tree Algorithms\\ 
4 & Shortest Path Algorithms \\
5 & Maximum Flow Algorithms\\
6 & Catch up \& Exam \\
\multicolumn{2}{l}{\textbf{\textit{Part 2: Modeling Various Applicatioons}}}\\
7 & Naval Applications \\
8 & Logistics Applications \\
9 & Social Network Applications \\
10 & Engineering Applications \\
11 & Linguistic Applications \\
12 & Catch up \& Exam \\
\multicolumn{2}{l}{\textbf{\textit{Part 3: Putting Everything Together}}}\\
13 & Group Project \\
14 & Group Project \\
15 & Group Project \\
16 & Group Project \& Review \\
\end{longtable}

\end{document}
