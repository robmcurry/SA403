% LaTeX Article Template
\documentclass[12pt]{article}
%% Other packages
\usepackage{amsmath}
\usepackage{amsthm}
\usepackage{titlesec}
\usepackage{soul}
\usepackage{tikz}
\usepackage{tikz-3dplot}
\usepackage{amssymb}
\usepackage{multicol}
\usepackage{float}
\usepackage{calc}
\usepackage{algorithm}
\usepackage{algorithmic}
\usepackage{fancybox}
\usepackage{array}
\usepackage[shortlabels]{enumitem}
\usepackage{framed}
\usepackage{hyperref}
\newcolumntype{L}[1]{>{\raggedright\let\newline\\\arraybackslash\hspace{0pt}}m{#1}}
\newcolumntype{C}[1]{>{\centering\let\newline\\\arraybackslash\hspace{0pt}}m{#1}}
\newcolumntype{R}[1]{>{\raggedleft\let\newline\\\arraybackslash\hspace{0pt}}m{#1}}


%% Margins
\usepackage{geometry}
\geometry{verbose,letterpaper,tmargin=1in,bmargin=1in,lmargin=1in,rmargin=1in}

\newcommand{\menuchoice}[2]{{\ttfamily#1..#2}}
\newcommand{\dotdot}{..}

\usepackage{graphicx}

% Array vertical and horizontal stretch
% \def\arraystretch{1.5}%  1 is the default, change whatever you need
% \setlength{\tabcolsep}{12pt}

%\graphicspath{%
\graphicspath{{./figs/}}

%% Paragraph style settings
\setlength{\parskip}{\medskipamount}
\setlength{\parindent}{0pt}

%% Change itemize bullets
\renewcommand{\labelitemi}{$\bullet$}
\renewcommand{\labelitemii}{$\circ$}
\renewcommand{\labelitemiii}{$\diamond$}
\renewcommand{\labelitemiv}{$\cdot$}

%% Shrink section fonts
\titleformat*{\section}{\large\bf}
\titleformat*{\subsection}{\normalsize\it}
\titleformat*{\subsubsection}{\normalsize\bf}

% %% Compress the spacing around section titles
\titlespacing*{\section}{0pt}{1.5ex}{0.75ex}
\titlespacing*{\subsection}{0pt}{1ex}{0.5ex}
\titlespacing*{\subsubsection}{0pt}{1ex}{0.5ex}

%% amsthm settings
\theoremstyle{definition}
\newtheorem{problem}{Problem}
\newtheorem{example}{Example}
\newtheorem{mydef}{Definition}

%% Answer box macros
%% \answerbox{alignment}{width}{height}
\newcommand{\answerbox}[3]{%
  \fbox{%
    \begin{minipage}[#1]{#2}
      \hfill\vspace{#3}
    \end{minipage}
  }
}

%% \answerboxfull{alignment}{height}
\newcommand{\answerboxfull}[2]{%
  \answerbox{#1}{\textwidth}{#2} 
}

%% \answerboxone{alignment}{height} -- for first-level bullet
\newcommand{\answerboxone}[2]{%
  \answerbox{#1}{6.15in}{#2} 
}

%% \answerboxtwo{alignment}{height} -- for second-level bullet
\newcommand{\answerboxtwo}[2]{%
  \answerbox{#1}{5.8in}{#2}
}

%% \graphbox{xmin}{xmax}{ymin}{ymax}{scale}
\newcommand{\graphbox}[5]%[-5, 5, -5, 5, 0.33]
{
\begin{tikzpicture}
     [>=latex,scale=#5]
     
     % Coordinate axes
     \draw [->,very thick] (#1, 0) -- (#2, 0) node[right] {$x$};
     \draw [->,very thick] (0, #3) -- (0, #4) node[above] {$y$};
     
     % Grid
     \draw[step=1cm,thick,dotted] (#1,#3) grid (#2,#4);
   \end{tikzpicture}
   }


%% Redefine maketitle
\makeatletter
\renewcommand{\maketitle}{
  \noindent SA403 -- Networks \\

  \begin{center}\Large{\textbf{\@title}}\end{center}
}
\makeatother

% Set the beginning of a LaTeX document
\begin{document}

%\graphbox{-10}{3}{-5}{10}

\title{Lesson 3: Algorithms}

%\graphbox[10][10]

\maketitle


\section*{Notes}

Book acknowledgment:
\begin{itemize}
	\item[]  ``Introduction to Algorithms" by Cormen, Leiserson, Rivest, and Stein. \emph{Third Edition}
\end{itemize}
\section*{Goals}
\begin{itemize}
\item Introduce definition, purpose, and outline of algorithms
\end{itemize}


\section{What is an Algorithm?}

How would you define an algorithm?

\vskip 4cm


\textbf{\emph{Definition}}:

``Informally, an \emph{algorithm} is any well-defined computational procedure that takes some value, or set of values, as \textbf{\emph{input}} and produces some value or set of values as an \textbf{\emph{output}}. An algorithm is thus a sequence of computational steps that transform the input into the output."

\vskip 0.5cm

What algorithms can you think of? 

\vskip 4cm

\newpage

An algorithm can also be viewed as a tool for solving a well-specified computational problem. The statement of the problem specifies in general terms the desired input/output relationship. The algorithm describes a specific computational procedure for achieving that input/output relationship.

\vskip 0.25cm

\textbf{\emph{Example:}}

Assume that you need to sort a sequence of numbers into non-decreasing order. This is commonly referred to as the Sorting Problem. 

What is the input for this problem?
\vskip 3cm
What is the output for this problem?
\vskip 3cm

%Input: A sequence of $n$ numbers $\langle a_1, a_2, \dots, a_n \rangle$.

%Output: A permutation or reordering $\langle a'_1, a'_2, \dots, a'_n \rangle$ of the input sequence such that $a'_1 \le a'_2 \le \dots a'_n$.


In your own words, what is an algorithm?
\vskip 3cm


What types of problems can be solved using algorithms?


\vskip 3cm
%\begin{itemize}
%	\item Sequencing the human genome
%	\item Think Facebook...
%	\item 
%\end{itemize}

\newpage
Parts of an algorithm:

\begin{itemize}
	\item Variables
	\begin{itemize}
		\item What values can change?
	\end{itemize}	
	\item Parameters
	\begin{itemize}
		\item What values must remain the same?
	\end{itemize}	
	\item Sequencing
	\begin{itemize}
		\item In what order show tasks be performed?
	\end{itemize}	
	\item Conditionals
	\begin{itemize}
		\item If $X$ occurs, then $Y$ must be performed.
	\begin{itemize}
		\item \emph{If}-statements
		\item \emph{While}-statements
		\item \emph{Do}-statements
	\end{itemize}
	\end{itemize}	

	\item Repetition
	\begin{itemize}
		\item When some set of tasks must be repeatedly performed.
	\end{itemize}
	\item Subroutines
	\begin{itemize}
		\item You can think of these as subtasks. 
	\end{itemize}
\end{itemize}

\newpage

\section{Try writing out an algorithm for solving the following problems...}
\subsection*{Tying your shoes}

\vskip 4cm

\subsection*{Finding a minimum value among a set of numbers}


\vskip 4cm

\subsection*{Service Selection Assignments}


\vskip 4cm
\newpage
\section{Algorithm Running Time}

The \emph{running time} of an algorithm on a particular input is the number of primitive operations or ``steps" executed. Assuming that each computer takes a constant amount of time to execute each line of code.  
\vskip 1cm
\textbf{Worst-case Running Time:}

The longest running time for \emph{any} input of size \emph{n}. This gives us an upper bound on the running time for any input. By knowing the worst-case running time, we can guarantee that the algorithm will never take any longer. 

\section{Written Steps}

Determine the maximum among values $x, \ y,$ and $z$.

\begin{enumerate}

\item If $x > y$ then proceed to step \ref{step1} Otherwise, proceed to step \ref{step2}.

\item If $x > z$ then return $x$ as the maximum value. Otherwise, return $z$ as the maximum value. \label{step1}

\item If $y > z$ then return $y$ as the maximum value. Otherwise, return $z$ as the maximum value. \label{step2}
\end{enumerate}

\section{Pseudocode}

You can think of pseudocode is a transition between writing out the steps and the code of an algorithm.

\begin{algorithm}
\caption{Determine maximum among values $x, \ y,$ and $z$}
\begin{algorithmic} 
%\REQUIRE $n \geq 0 \vee x \neq 0$
%\ENSURE $y = x^n$
%\STATE $y \leftarrow 1$
\IF{$x > y$}
	\IF{$x>z$}
		\STATE Maximum value $= x$.
	\ELSE
		\STATE Maximum value $= z$.
	\ENDIF
\ELSE
	\IF{$y>z$}
		\STATE Maximum value $= y$.
	\ELSE
		\STATE Maximum value $= z$.
	\ENDIF
\ENDIF
\end{algorithmic}
\end{algorithm}

%\begin{algorithm}
%\caption{Determine maximum among values $x, \ y,$ and $z$}
%\begin{algorithmic} 
%\REQUIRE $n \geq 0 \vee x \neq 0$
%\ENSURE $y = x^n$
%\STATE $y \leftarrow 1$
%\IF{$x > y$}
%	\IF{$x>z$}
%		\STATE Maximum value $= x$.
%	\ELSE
%		\STATE Maximum value $= z$.
%	\ENDIF
%\ELSE
%	\IF{$y>z$}
%		\STATE Maximum value $= y$.
%	\ELSE
%		\STATE Maximum value $= z$.
%	\ENDIF
%\ENDIF
%\end{algorithmic}
%\end{algorithm}

\end{document}
